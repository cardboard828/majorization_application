

物理的に実現可能な量子操作は次の写像で表現されると考える. 

\begin{mydfn}[CP写像]\label{dfn.CP-map}
  $\symcal{E}:\symcal{L}(\symcal{H})\to\symcal{L}(\symcal{H}')$は線型写像とする. 
  $\symcal{E}$が以下の3つを満たすとき, これをCP写像と呼ぶ(Complete-Positive). 
  \begin{enumerate}
    \item $X$, $Y$を演算子とする. 
    このとき, $X\geq 0$ならば$\symcal{E}(X)\geq 0$. 
    この条件を満たせば, その写像はPositiveであるという. 
    \item $n$次元Hilbert空間を$\symcal{H}_n$として, $\symcal{I}_n$を$\symcal{L}(\symcal{H}_n)$上の恒等演算子とする. 
    このとき, $\symcal{E}\otimes \symcal{I}_n$がPositiveとなる. 
    これを$n$-positiveと呼ぶ. 
    \item 任意の$n\in \N$について, $\symcal{E}$は$n$-positive. 
  \end{enumerate}
\end{mydfn}

参考のために, PositiveだがCPではない写像の例を挙げる. \\

\begin{e.g.}[Positive but not CP]
  簡単のためにqubit空間$\symcal{H}_2$上で考える. 
  いつも通り$\{\ket{0}, \ket{1}\}$を正規直交基底とする. 
  $\symcal{E}_T:\symcal{L}(\symcal{H}_2)\to \symcal{L}(\symcal{H}_2)$を$\symcal{E}_T:\ket{i}\bra{j}\mapsto \ket{j}\bra{i}$な線型写像とする. 
  これはPositiveである\footnote{$\begin{pmatrix}\alpha^*&\beta^*\end{pmatrix}\begin{pmatrix}a&b\\c&d\end{pmatrix}\begin{pmatrix}\alpha\\ \beta\end{pmatrix}\geq 0$であるとき両辺転置を取れば$\begin{pmatrix}\alpha&\beta\end{pmatrix}\begin{pmatrix}a&c\\b&d\end{pmatrix}\begin{pmatrix}\alpha^*\\ \beta^*\end{pmatrix}\geq 0$}. 
  次に, $I\otimes \symcal{E}_T$を例えば$\ket{\psi}=\ket{00}+\ket{11}$(規格化は無視)に作用させることを考える. 
  \begin{equation}
    \rho_{\psi}=\ket{0}\bra{0}\otimes \ket{0}\bra{0}+\ket{0}\bra{1}\otimes \ket{0}\bra{1}+\ket{1}\bra{0}\otimes \ket{1}\bra{0}+\ket{1}\bra{1}\otimes \ket{1}\bra{1}
  \end{equation}
  とかけることに注意すれば, 
  \begin{align}
    I\otimes \symcal{E}_T(\rho_{\psi})&=\ket{0}\bra{0}\otimes \ket{0}\bra{0}+\ket{0}\bra{1}\otimes \ket{1}\bra{0}+\ket{1}\bra{0}\otimes \ket{0}\bra{1}+\ket{1}\bra{1}\otimes \ket{1}\bra{1}\\
    &\stackrel{\cdot}{=}\begin{pmatrix}
      1 & & & \\
       & 0&1& \\
       & 1&0& \\
       &  & &1 \\
    \end{pmatrix}
  \end{align}
  となる. 
  この行列の固有値は$1,1,1,-1$なので, 確かにこれはCPではない. \\
\end{e.g.}
次にTP(Trace Preserving)を定義する. 

\begin{mydfn}[TP写像]\label{dfn.TP-map}
  線型写像$\symcal{E}$が任意の演算子$X$について
  \begin{equation}
    \tr(\symcal{E}(X))=\tr(X)
  \end{equation}
  を満たすとき, これをTPという. 
  また, 任意の$X\geq 0$について$\tr(\symcal{E}(X))\leq\tr(X)$のとき, $\symcal{E}$はtrace-nonincreasingという. 
\end{mydfn}

CPかつTPな写像のことをCPTP写像と呼ぶ. 
CPTP写像には, 理解が深まる同値な表現がある. 

\begin{mythm}[CPTP写像の同値な表現]\label{thm.CPTP-map}
  線型写像$\symcal{E}$について, 以下は同値:
  \begin{enumerate}[(a). ]
    \item $\symcal{E}$はCPTP写像. 
    \item $\symcal{E}$は$\sum_{k}M_k^{\dag}M_k=I$を満たすKraus演算子で$\symcal{E}(\rho)=\sum_{k}M_k\rho M_k^{\dag}$とかける. 
    \item 環境系を表す量子状態$\sigma\in\symcal{S}(\symcal{H}_e)$が存在して, あるUnitary演算子$U:\symcal{H}\otimes \symcal{H}_e\to \symcal{H}'\otimes \symcal{H}_e'$を使って
      $\symcal{E}(\rho)=\tr_{e'}(U\rho\otimes\sigma U^{\dag})$
    とかける. 
    これをStinespring表現という. 
  \end{enumerate}
\end{mythm}

\begin{proof}
  Appendixにかけたら書く. 
\end{proof}

(b)は使い勝手の良い性質で, 写像がKraus演算子を使ってかけていたらそれは量子操作を記述するCPTP写像になっているとわかる. 
(c)は物理的な意味をより明確にしてくれていて, 結局視野を環境系にまで広げれば知っているようにユニタリ時間発展をしていて, CPTP写像は環境系をトレースアウトした着目系の間の時間発展の記述をしているとわかる. \\

\begin{e.g.}
  例えば, 部分トレースはCPTP写像になっている. 
  部分トレースの定義\ref{dfn.partial_trace}の文字を流用する. 
  $\symcal{H}_B$の正規直交基底を$\{\ket{b_i}\}$と書くことにして, $I$を$\symcal{H}_A$上の恒等演算子とする. 
  $M_i\coloneqq I\otimes \bra{b_i}$とすると, $M_i^\dag=I\otimes \ket{b_i}$なのでこれはKraus演算子になっていて, 対応するCPTP写像は部分トレースそのものである. 
  部分トレースがCPTP写像になっていることは後で使う. 
\end{e.g.}