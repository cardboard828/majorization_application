この章では\ref{chap.quantum_info_basic}章で導入した概念を使って, 量子熱力学の文脈でのSingle-Shotの変換理論の話をする. 
Single-Shotの変換理論とは, 「ある状態を持っていて, それを別の状態に, 特定のクラスの操作で変換できるかどうか」を考える理論のことを指す(と思う)
\footnote{田﨑熱力学で強調されていることだが, (古典)熱力学においてエントロピーは状態間の断熱変化可能性を完全に特徴づけていた. 
すなわち, 断熱操作の元で系のエントロピーは増加するし, エントロピーが増加するような状態であればうまい断熱操作が存在してその状態を作ることができる. }. 
この問題を考えることで, 状態遷移に必要な仕事の下界や, 遷移に伴って得られるエネルギーの上界を得ることができる. 