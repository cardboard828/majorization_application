導入のために, 二つの直交しているとは限らない状態$\ket{a}$, $\ket{b}$を識別する問題を考えよう. 
測定を使って初めて量子状態の持つ情報にアクセスできるので, 測定することで識別しようとする. 

\begin{tcolorbox}[colback=black!5!white,colframe=black!75!white]
    POVMを定義した後の例で触れた二値POVMを使って, $Q$が測定されたら量子状態は$\ket{a}$, $I-Q$が測定されたら$\ket{b}$と判断することにしよう.
    上手いPOVMが存在して, 二つの状態は必ず識別可能か?
    すなわち, $\bra{a}Q\ket{a}=1$, $\bra{b}(I-Q)\ket{b}=1$なる$0< Q < I$はあるか?
\end{tcolorbox}

\begin{proof}[検証]   
    存在すると仮定. 
    $\bra{a}I\ket{a}=1$より$\bra{a}(I-Q)\ket{a}=0$. 
    従って$\sqrt{I-Q}\ket{a}=0$. 
    $\ket{b}=\alpha\ket{a}+\gamma\ket{c}$($\ket{c}$は$\ket{a}$と直交する規格化されたベクトル)と$\ket{b}$を展開する. 
    $\ket{a}, \ket{b}$も規格化されているので$|\alpha|^2+|\gamma|^2=1$. 
    条件の$\bra{b}(I-Q)\ket{b}=1$より$1=\bra{b}\sqrt{I-Q}\sqrt{I-Q}(\alpha\ket{a}+\gamma\ket{c})=|\gamma|^2$となるが, 規格化条件より$\alpha=0$のときのみ成立. 
    よって$\ket{a}, \ket{b}$が直交しているときは良いPOVMはあるが, それ以外の場合は誤差なし識別は不可能. 
\end{proof}

つまり, 用意された二つの状態すら一般には完全な識別は不可能!
これは古典にはない量子の非自明な部分だ. 
この章では一番基本的な, 用意された二つの状態を識別する, というタスクについて少し掘り下げる. 