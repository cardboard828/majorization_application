
まず量子状態を線型代数の言葉を使って表すことから始める. 
量子システムはHilbert空間$\symcal{H}$で記述される\footnote{内積の入った複素数上の線型空間と思ってもらって大丈夫だと思う...}. 
$\symcal{H}$に作用する演算子の集合を$\symcal{L}(\symcal{H})$と書く. 
$d$を$\symcal{H}$の次元とする. 

% 系の\textbf{純粋状態}は以下のように定義される. 
% \begin{mydfn}[純粋状態]
%   系の純粋状態は$\braket{\phi|\phi}=1$と規格化されたベクトル$\ket{\phi}\in\symcal{H}$で表される. 
% \end{mydfn}

一般の量子状態は以下で定める\textbf{密度演算子}によって表される. 
\begin{mydfn}[量子状態]\label{dfn.quantum_states}
  系の量子状態は$\symcal{H}$に作用する$\tr[\rho]=1$かつ半正定値な演算子$\rho\in\symcal{L}(\symcal{H})$によって表され, $\rho$を密度演算子と呼ぶ. 
\end{mydfn}
ここで半正定値とは任意のベクトル$\ket{\psi}\in\symcal{H}$について$\bra{\psi}\rho\ket{\psi}\geq 0$が成立することを意味し, これを$\rho\geq 0$と表す. 

$\symcal{H}$上の量子状態の集合を$\symcal{S}(\symcal{\rho})$と表すことにする. 
これは凸集合になっている. 
\begin{myprop}
  $\symcal{S}(\symcal{H})$は凸集合. 
\end{myprop}
\begin{proof}
  $p\in [0,1]$, $\rho, \sigma\in\symcal{S}(\symcal{H})$とする. 
  トレースの線形性から$\tr[p\rho+(1-p)\sigma]=p\tr[\rho]+(1-p)\tr[\sigma]=1$. 
  また, 内積の線型性から$\bra{\psi}(p\rho+(1-p)\sigma)\ket{\psi}=p\bra{\psi}\rho\ket{\psi}+(1-p)\bra{\psi}\sigma\ket{\psi}\geq 0$. 
\end{proof}
結局, 状態を表す密度演算子の確率混合もちゃんと状態になるようになっている. 

次に以下のように純粋状態と混合状態という概念を導入する. 
\begin{mydfn}[純粋状態, 混合状態]
  $\symcal{S}(\symcal{H})$の端点を純粋状態, そうでない元を混合状態と呼ぶ. 
  ここで$\rho\in\symcal{S}(\symcal{H})$が端点であるとは, これが$\symcal{S}(\symcal{H})$の他の元の非自明な凸結合で表せないことをいう. 
\end{mydfn}

この定義の意味に触れる前に, 密度演算子がスペクトル分解できることに触れる. 
\begin{myprop}[スペクトル分解]
  密度演算子$\rho\in\symcal{S}(\symcal{H})$は$\symcal{H}$を張る正規直交系$\{\ket{\phi}_i\}$で以下のように対角化できる. 
  \begin{equation}
    \rho=\sum_{i=1}^d p_i\ket{\phi_i}\bra{\phi_i}
  \end{equation}
  ここで$p_i\geq 0$で, $\ket{\phi_i}\bra{\phi_i}$は射影演算子である. 
\end{myprop}
\begin{proof}
  \cite{nielsen2010quantum}の2章を参照. 
\end{proof}

この命題から, 純粋状態はあるベクトル$\ket{\phi}$で貼られる1次元部分空間への射影演算子だとわかる. 
また, 1次元部分空間への射影演算子は$\symcal{S}(\symcal{H})$の非自明な凸結合で書けないので, 結局純粋状態は適当な$\ket{\psi}\in\symcal{H}$で指定できる. 
トレースが1である条件も加味して純粋状態を$\braket{\psi|\psi}=1$と規格化されたベクトル$\ket{\psi}\in\symcal{H}$で表すことも多い
\footnote{というか量子力学というとほぼこのベクトルで表される状態を扱っている. }. 

また, 混合状態はある純粋状態の集まりを古典確率混合したものだと捉えることができる. 
測定の話をした後に簡単な例を紹介する. 

% しかし注意すべきは, ある混合状態に対して, その状態を作る純粋状態の集まりと確率分布の組は複数あっても良いということだ. 
% 例えば密度行列として(規格化した)単位行列をとってきた場合を考えれば良い. 




