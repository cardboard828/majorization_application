

次に測定という操作を定義する. 
\begin{mydfn}[測定]
  量子測定は$\symcal{H}$上のオペレータの集合$\{M_m\}$のうち, $\sum_{m}M_m^\dag M_m=I$を満たすものにより記述される. 
  ここで$m$は測定結果を指定するラベルとする. 
  量子状態$\rho\in\symcal{S}(\symcal{H})$について$\{M_m\}$で記述される測定を行うとき, 結果$m$が得られる確率$p(m)$は
  \begin{equation}
    p(m)=\tr[M_m^\dag M_m\rho]
  \end{equation}
  で与えられ, 測定後の状態$\rho_m$は
  \begin{equation}
    \rho_m=\frac{M_m\rho M_m^\dag}{\tr[M_m^\dag M_m\rho]}
  \end{equation}
  となる. 
\end{mydfn}
測定と混合状態の例を紹介する. 
\begin{e.g.}[Stern-Gerlach実験]
  2次元の状態空間$\symcal{H}_2$で記述される量子系を考える
  \footnote{これをqubitの系といったりする}. 
  $\{\ket{0}, \ket{1}\}$は$\symcal{H}_2$を張る正規直交基底とする. 
  始状態として$\ket{0}$を用意する. 
  また, $\ket{\uparrow}\coloneqq (\ket{0}+\ket{1})/\sqrt{2}$, $\ket{\downarrow}\coloneqq (\ket{0}-\ket{1})/\sqrt{2}$として, $\{\ket{\uparrow}\bra{\uparrow}, \ket{\downarrow}\bra{\downarrow}\}$で記述される測定をこの系に行う. 
  計算すると(計算しなくともこの実験を知っている人ならわかることだが), 1/2の確率でそれぞれ$\uparrow$, $\downarrow$が測定され, 測定後の状態はそれぞれ$\ket{\uparrow}$, $\ket{\downarrow}$になっている. 
  そのアンサンブル平均状態$\rho$は
  \begin{align}
    \rho&=1/2\ket{\uparrow}\bra{\uparrow}+1/2\ket{\downarrow}\bra{\downarrow}\\
    &=\tr[(\ket{\uparrow}\bra{\uparrow})\ket{0}\bra{0}]\frac{(\ket{\uparrow}\bra{\uparrow})\ket{0}\bra{0}(\ket{\uparrow}\bra{\uparrow})}{\tr[(\ket{\uparrow}\bra{\uparrow})\ket{0}\bra{0}]}+\tr[(\ket{\downarrow}\bra{\downarrow})\ket{0}\bra{0}]\frac{(\ket{\downarrow}\bra{\downarrow})\ket{0}\bra{0}(\ket{\downarrow}\bra{\downarrow})}{\tr[(\ket{\downarrow}\bra{\downarrow})\ket{0}\bra{0}]}\\
    &=(\ket{\uparrow}\bra{\uparrow})\ket{0}\bra{0}(\ket{\uparrow}\bra{\uparrow})+(\ket{\downarrow}\bra{\downarrow})\ket{0}\bra{0}(\ket{\downarrow}\bra{\downarrow})
  \end{align}
  と記述でき, これはKraus表示になっているため, 測定のアンサンブル状態\footnote{とか, 測定はされたが結果を知らない状態とか}は測定演算子をKraus演算子とするCPTP写像で与えられるとわかる. 
  ちなみにStern-Gerlach実験はこの後特に$\uparrow$が測定された状態のみを抽出し, $\{\ket{0}\bra{0}, \ket{1}\bra{1}\}$で記述される測定を行う. 
  同様に考えれば, 1/2の確率でそれぞれ0, 1が測定されると分かる
  \footnote{量子論黎明期にこのような実験が実際に行われ結果が論理的に考察されたおかげで今展開しているような量子論の枠組みがある. }. 
\end{e.g.}

