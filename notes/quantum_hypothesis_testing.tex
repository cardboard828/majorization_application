

2つの状態$\rho, \sigma\in\symcal{S}(\symcal{H})$の識別をPOVM$\{Q, (I-Q)\}$ですることを考える. 
$Q$の測定値を0, $I-Q$の測定値を1とする. 
$\rho$を帰無仮説, $\sigma$を対立仮説という. 
測定値が0のとき$\rho$が真, 測定値が1のとき$\sigma$が真という. 
前述のように完全な識別はできないので, それぞれ誤りがある. 
本当は$\rho$が用意されたのに誤って$\sigma$が用意されたと判断する確率は
\begin{equation}
  \alpha(Q)\coloneqq\tr[(I-Q)\rho]
\end{equation}
で与えられ, これを\textbf{第一種誤り確率}という. 
同じように考えて, 本当は$\sigma$が用意されたのに誤って$\rho$が用意されたと判断する確率は
\begin{equation}
  \beta(Q)\coloneqq \tr[Q\sigma]
\end{equation}
で与えられ, これを\textbf{第二種誤り確率}という. \\

$\rho$が来たときに$\rho$と判断する成功確率$\tr[Q\rho]$を$0<\eta<1$以上にするという条件のもと, 第二種誤り確率可能な限り小さくする問題を考える. 
このとき, 達成された最小値の対数をとり, これを量子仮説検定ダイバージェンスと呼ぶ. 
\begin{mydfn}[量子仮説検定ダイバージェンス]
  $0<\eta<1$ について量子仮説検定ダイバージェンス$S_{\text{H}}^\eta(\rho||\sigma)$を次のように定義する. 
  \begin{equation}
    S_{\text{H}}^\eta(\rho||\sigma)\coloneqq -\ln\left( \frac{1}{\eta}\min_{0\leq Q\leq I, \tr[\rho Q]\geq \eta} \tr[\sigma Q]\right)
  \end{equation}
\end{mydfn}

詳細は省略するが, この仮説検定ダイバージェンスは定義\ref{dfn.smoothed_Renyi-alpha-div}で定義した平滑化したダイバージェンスと関連がある. 
\begin{equation}
  S_{\text{{H}}}^{\eta\simeq 1}(\rho||\sigma)\simeq S_0^{\varepsilon\simeq 0}(\rho||\sigma),\quad S_{\text{{H}}}^{\eta\simeq 0}(\rho||\sigma)\simeq S_\infty^{\varepsilon\simeq 0}(\rho||\sigma). 
\end{equation}

% \begin{table}[H]
%   \centering\begin{tabular}{ccccc}
%     用意された状態&測定値&確率&判定&正誤\\
%     \hline
%     $\rho$&\begin{tabular}{cccc}
%       0&$\tr[Q\rho]$&$\rho$&正\\
%       \hline
%       1&$\tr[(I-Q)\rho]$&$\sigma$&第一種誤り
%     \end{tabular}
%   \end{tabular}
%   \caption{量子仮説検定における誤り}
%   \label{tab.hypothesis_testing}
% \end{table}