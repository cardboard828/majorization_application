
この節では後の章で登場する色々なエントロピーを定義する. 
情報理論では情報源の持つ平均情報量を定量化するShannonエントロピーが本質的に重要で, そこから相対エントロピー(KLダイバージェンス)等の概念が生まれたが, 以下のエントロピーはそれらを量子論に拡張したものになっている. 

\begin{mydfn}[von Neumann エントロピー]\label{dfn.vonNeumannEntropy}
  次元$d$のvon Neumann エントロピー$S_1(\rho)$は次のように定義される. 
  \begin{equation}
    S_1(\rho)\coloneqq -\tr(\rho\ln{\rho})
  \end{equation}
\end{mydfn}

\begin{mydfn}[量子KLダイバージェンス]\label{dfn.KLdiv}
  次元$d$の量子KLダイバージェンス$S_1(\rho||\sigma)$は次のように定義される. 
  \begin{equation}
    S_1(\rho||\sigma)\coloneqq \tr(\rho\ln{\rho}-\rho\ln{\sigma})
  \end{equation}
  ここで$\rho$の台は$\sigma$のそれに包含されていると考える. 
  すなわち, $\Ker{\sigma}\subset \Ker{\rho}$とする.
\end{mydfn}