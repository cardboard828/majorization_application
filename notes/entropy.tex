
この節では後の章で登場する色々なエントロピーを定義する. 
情報理論では情報源の持つ平均情報量を定量化するShannonエントロピーが本質的に重要で, そこから相対エントロピー(KLダイバージェンス)等の概念が生まれたが, 以下のエントロピーはそれらを量子論に拡張したものになっている. 
以下では$\rho, \sigma\in\symcal{S}(\symcal{H})$とする. 
なお, $\rho$の台は常に$\sigma$の台に含まれているものとする. 



\begin{mydfn}[von Neumann エントロピー]\label{dfn.vonNeumannEntropy}
  von Neumann エントロピー$S_1(\rho)$は次のように定義される. 
  \begin{equation}
    S_1(\rho)\coloneqq -\tr(\rho\ln{\rho})
  \end{equation}
\end{mydfn}

\begin{mydfn}[量子KLダイバージェンス]\label{dfn.KLdiv}
  量子KLダイバージェンス$S_1(\rho||\sigma)$は次のように定義される. 
  \begin{equation}
    S_1(\rho||\sigma)\coloneqq \tr(\rho\ln{\rho}-\rho\ln{\sigma})
  \end{equation}
  ここで$\rho$の台は$\sigma$のそれに包含されていると考える. 
  すなわち, $\Ker{\sigma}\subset \Ker{\rho}$とする.
\end{mydfn}

\begin{mydfn}[R\'{e}nyi-αダイバージェンス]\label{dfn.Renyi-alpha-div}
  $\alpha=0, \infty$について, R\'{e}nyi-αダイバージェンス$S_{\alpha}(\rho||\sigma)$を以下のように定義する. 
  \begin{equation}
    S_{0}(\rho||\sigma)\coloneqq -\ln\left( \tr[P_{\rho}\sigma] \right), \quad S_{\infty}(\rho||\sigma)\coloneqq \ln\left( \min\{\lambda: \rho\leq \lambda\sigma\} \right)
  \end{equation}
  ここで$P_{\rho}$は$\rho$の台への射影演算子である. 
\end{mydfn}

この$\alpha=0, \infty$のR\'{e}nyi-αダイバージェンスについて, 以下の定理が知られている. 

\begin{mythm}\label{thm.Renyi-alpha-div_alpha-increasing}
  \begin{equation}
    0\leq S_0(\rho||\sigma)\leq S_1(\rho||\sigma)\leq S_\infty(\rho||\sigma)
  \end{equation}
  ここで, $S_0(\rho||\sigma)=0$が成立$\iff$ $\rho$と$\sigma$の台が等しいときである. 
  また, $S_1(\rho||\sigma)=S_\infty(\rho||\sigma)=0$が成立$\iff$ $\rho=\sigma$のとき. 
\end{mythm}

\begin{proof}
  前半の不等式は\cite{SagawaEntropy}を参照. 
  後半の等号成立条件を検討する. 
  この説の仮定$\Ker{\sigma}\leq\Ker{\rho}$より, $S_0(\rho||\sigma)=0\iff \tr[P_{\rho}\sigma]=1\iff \Ker{\sigma}=\Ker{\rho}$, よって一つ目の条件は示された. 
  二つ目について. 
  $\rho=\sigma$のとき, 定義\ref{dfn.Renyi-alpha-div}より$S_{\infty}(\rho||\sigma)=0$であり, かつ$\rho$と$\sigma$の台が等しい, つまり$S_0(\rho||\sigma)=0$なので, 不等式より$S_1(\rho||\sigma)=0$. 
  また, $S_{\infty}(\rho||\sigma)=0$のとき, つまり$\rho\leq\sigma$な訳だが, どちらもトレース1なので, $\rho=\sigma$となるしかない. 
\end{proof}

\begin{mydfn}[平滑化R\'{e}nyi 0/∞-ダイバージェンス]\label{dfn.smoothed_Renyi-alpha-div}
  $\rho\in\symcal{S}(\symcal{H})$の$\varepsilon$近傍を
  \begin{equation}
    B^{\varepsilon}(\rho)\coloneqq \{\tau: D(\tau, \rho)\leq \varepsilon, \tr[\tau]=1, \tau\geq 0\}
  \end{equation}
  とする. 
  これを使って, 平滑化R\'{e}nyi 0/∞-ダイバージェンスは以下のように定義される. 
  \begin{align}
    S_{\infty}^{\varepsilon}(\rho||\sigma)&\coloneqq \min_{\tau\in B^{\varepsilon}(\rho)}S_{\infty}(\tau||\sigma), \label{dfn.smoothed_Renyi_infty_div}\\
    S_{0}^{\varepsilon}(\rho||\sigma)&\coloneqq \max_{\tau\in B^{\varepsilon}(\rho)}S_{0}(\tau||\sigma). \label{dfn.smoothed_Renyi_0_div}
  \end{align}
\end{mydfn}

${\rho}_n, \sigma_n\in\symcal{S}(\symcal{H}^{\otimes n})$の列を$\widehat{P}\coloneqq \{{\rho}_{n}\}_{n\in\N}$, $\widehat{\Sigma}\coloneqq \{\sigma_{n}\}_{n\in\N}$とする.
上で定義したKLダイバージェンス, 平滑化ダイバージェンスを使って, 以下のようなダイバージェンス・レートを定義する. 

\begin{mydfn}[Quantum spectral divergence rates]\label{dfn.Quantum_spectral_divergence_rates}
  KLダイバージェンス・レートを以下のように定義する. 
  \begin{equation}
    S_1(\widehat{P}||\widehat{\Sigma})\coloneqq \lim_{n\to\infty}\frac{1}{n}S_1(\rho_n||\sigma_n)
  \end{equation}
  なお, この極限が必ず存在するとは限らない. 
  upperダイバージェンス・レート, lowerダイバージェンス・レートはそれぞれ以下のように定義される. 
  \begin{align}
    \bar{S}(\widehat{P}||\widehat{\Sigma})&\coloneqq \lim_{\varepsilon\to +0}\limsup_{n\to\infty}\frac{1}{n}S_{\infty}^{\varepsilon}(\rho_n||\sigma_n)\\
    \underline{S}(\widehat{P}||\widehat{\Sigma})&\coloneqq \lim_{\varepsilon\to +0}\liminf_{n\to\infty}\frac{1}{n}S_{0}^{\varepsilon}(\rho_n||\sigma_n)
  \end{align}
\end{mydfn}


