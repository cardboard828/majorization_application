\documentclass[report]{jlreq}
\usepackage{luatexja}
\usepackage{fontspec}
\usepackage{import}
\subimport{../preamble}{preamble}
  
\title{量子情報入門からSingle-Shotの量子熱力学まで}
\author{森}
\date{\today}
  
\begin{document}
\maketitle

\setcounter{tocdepth}{1}
\thispagestyle{TOC}
\tableofcontents


\pagestyle{mystyle}
\chapter{量子情報の枠組みミニマム}
まず量子情報分野の基本的なルールについて, 以降の章で使うものについて簡単な説明をする. 
基本的に有限次元の量子力学が舞台になっている. 
主に\cite{nielsen2010quantum}, \cite{SagawaEntropy}を参考にした. 
怪しげなことを書いていたら教えて欲しいです. 
あと「演算子」, 「オペレータ」は適宜「行列」にでも読み替えてもらって大丈夫です. 
\section{状態}
\subimport{notes/}{states.tex}
\section{合成系}
\subimport{notes/}{composite_system.tex}
\section{量子操作}
\subimport{notes/}{CPTP_maps.tex}
\section{測定}
\subimport{notes/}{measurement.tex}
\section{量子状態間の尺度}
\subsection{トレース距離}
\subsection{忠実度}
\subsection{応用例:クローン不可能定理}
\section{色々なエントロピー}
\subsection{von Neumannエントロピー}
\subsection{量子相対エントロピー}
\subsection{R\'{e}nyi-αダイバージェンス}
\subsection{平滑化R\'{e}nyi-αダイバージェンス}

\chapter{量子仮説検定}
\section{状況設定}
\section{量子Neyman-Pearson検定}
\section{漸近論と量子相対エントロピーの創発}

\chapter{Single-Shotの量子熱力学}
\section{状況設定}
\section{Exact Case}
\section{Approximate Case}
\section{Asymptotic Case}

\appendix
\chapter{CPTP写像の3つの表現の同値性の証明}
\chapter{i.i.d.での量子Steinの補題の証明}
  
\begin{thebibliography}{99}
\bibitem{nielsen2010quantum}Michael A Nielsen and Isaac L Chuang. 
    \textit{Quantum computation and quantum information}.
    Cambridge university press, 2010.
\bibitem{SagawaEntropy}Takahiro Sagawa. 
    \textit{Entropy, Divergence, and Majorization in Classical and Quantum Thermodynamics}. 
    Springer Singapore, 2022.
\end{thebibliography}
  
\end{document}