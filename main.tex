\documentclass[report]{jlreq}
\usepackage{luatexja}
\usepackage{fontspec}
\usepackage{import}
\subimport{../preamble}{preamble}
  
\title{量子情報入門からSingle-Shotの量子熱力学まで}
\author{森}
\date{\today}
  
\begin{document}
\maketitle

\setcounter{tocdepth}{1}
\thispagestyle{TOC}
\tableofcontents

\chapter{量子情報の枠組みミニマム}
\section{状態}
\subsection{純粋状態}
\subsection{混合状態}
\section{量子操作}
\subimport{notes/}{CPTP_maps.tex}
\subsection{モチベーション}
\subsection{同値な表現}
\section{測定}
\subsection{量子測定}
\subsection{POVM}
\subsection{応用例:量子状態の識別}
\section{複合系}
\section{量子状態間の尺度}
\subsection{トレース距離}
\subsection{忠実度}
\subsection{応用例:クローン不可能定理}
\section{色々なエントロピー}
\subsection{von Neumannエントロピー}
\subsection{量子相対エントロピー}
\subsection{R\'{e}nyi-αダイバージェンス}
\subsection{平滑化R\'{e}nyi-αダイバージェンス}

\chapter{量子仮説検定}
\section{状況設定}
\section{量子Neyman-Pearson検定}
\section{漸近論と量子相対エントロピーの創発}

\chapter{Single-Shotの量子熱力学}
\section{状況設定}
\section{Exact Case}
\section{Approximate Case}
\section{Assymptotic Case}

\appendix
\chapter{CPTP写像の3つの表現の同値性の証明}
\chapter{i.i.d.での量子Steinの補題の証明}



\subimport{notes}{sample.tex}
  
\begin{thebibliography}{99}
    \bibitem{sample}サンプル
\end{thebibliography}
  
\end{document}