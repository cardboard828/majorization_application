\documentclass[report]{jlreq}
\usepackage{luatexja}
\usepackage{fontspec}
\usepackage{import}
\subimport{../preamble}{preamble}
  
\title{量子情報入門からSingle-Shotの量子熱力学まで}
\author{森}
\date{\today}
  
\subimport{settings/}{bibliography_settings.tex}

\begin{document}
\maketitle

\setcounter{tocdepth}{1}
\thispagestyle{TOC}
\tableofcontents

\pagestyle{mystyle}

\chapter{量子情報の枠組みミニマム}\label{chap.quantum_info_basic}
\subimport{notes/}{intro_chap1.tex}
\section{状態}
\subimport{notes/}{states.tex}
\section{合成系}
\subimport{notes/}{composite_system.tex}
\section{量子操作}
\subimport{notes/}{CPTP_maps.tex}
\section{測定}
\subimport{notes/}{measurement.tex}
\section{量子状態間の尺度}
\subimport{notes/}{distance.tex}
% \subsection{忠実度}
% \subsection{応用例:クローン不可能定理}
\section{色々なエントロピー}\label{sec.entropy}
\subimport{notes/}{entropy.tex}

\chapter{量子仮説検定}
\subimport{notes/}{intro_chap2.tex}

\section{状況設定}
\subimport{notes/}{quantum_hypothesis_testing.tex}
% \section{量子Neyman-Pearson検定}
\section{漸近論と量子Steinの補題}
\subimport{notes/}{stein's_lemma.tex}

\chapter{Single-Shotの量子熱力学}
\subimport{notes/}{intro_chap3.tex}

\section{状況設定}\label{sec.Single-Shot_quantumthermo}
\subimport{notes/}{Single-Shot_quantumthermo.tex}
\section{Exact Case}
\subimport{notes/}{exact_case.tex}
\section{Approximate Case}
\subimport{notes/}{approximate_case.tex}
\section{Asymptotic Case}
\subimport{notes/}{asymptotic_case.tex}


\pagestyle{appendix}

\appendix
\chapter{CPTP写像の3つの表現の同値性の証明}
\subimport{notes/}{proof_on_Kraus-op_Stinespring-representation.tex}
  
\pagestyle{TOC}
\subimport{notes/}{sankobunken.tex}
  
\end{document}