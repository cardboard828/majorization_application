\documentclass[report]{jlreq}
\usepackage{luatexja}
\usepackage{fontspec}
\usepackage{import}
\subimport{../preamble}{preamble}
  
\title{量子情報入門からSingle-Shotの量子熱力学まで}
\author{森}
\date{\today}
  
\begin{document}
\maketitle

\setcounter{tocdepth}{1}
\thispagestyle{TOC}
\tableofcontents


\pagestyle{mystyle}
\chapter{量子情報の枠組みミニマム}\label{chap.quantum_info_basic}
まず量子情報分野の基本的なルールについて, 以降の章で使うものについて簡単な説明をする. 
基本的に有限次元の量子力学が舞台になっている. 
主に\cite{nielsen2010quantum}, \cite{SagawaEntropy}を参考にした. 
怪しげなことを書いていたら教えて欲しいです. 
あと「演算子」, 「オペレータ」は適宜「行列」にでも読み替えてもらって大丈夫です. 
\section{状態}
\subimport{notes/}{states.tex}
\section{合成系}
\subimport{notes/}{composite_system.tex}
\section{量子操作}
\subimport{notes/}{CPTP_maps.tex}
\section{測定}
\subimport{notes/}{measurement.tex}
\section{量子状態間の尺度}
\subimport{notes/}{distance.tex}
% \subsection{忠実度}
% \subsection{応用例:クローン不可能定理}
\section{色々なエントロピー}
\subimport{notes/}{entropy.tex}

% \chapter{量子仮説検定}
% \section{状況設定}
% \section{量子Neyman-Pearson検定}
% \section{漸近論と量子相対エントロピー}

\chapter{Single-Shotの量子熱力学}
この章では\ref{chap.quantum_info_basic}章で導入した概念を使って, 量子熱力学の文脈でのSingle-Shotの変換理論の話をする. 
Single-Shotの変換理論とは, 「ある状態を持っていて, それを別の状態に, 特定のクラスの操作で変換できるかどうか」を考える理論のことを指す(と思う)
\footnote{田﨑熱力学で強調されていることだが, (古典)熱力学においてエントロピーは状態間の断熱変化可能性を完全に特徴づけていた. 
すなわち, 断熱操作の元で系のエントロピーは増加するし, エントロピーが増加するような状態であればうまい断熱操作が存在してその状態を作ることができる. }. 
この問題を考えることで, 状態遷移に必要な仕事の下界や, 遷移に伴って得られるエネルギーの上界を得ることができる. 

\section{状況設定}\label{sec.Single-Shot_quantumthermo}
\subimport{notes/}{Single-Shot_quantumthermo.tex}
\section{Exact Case}
\subimport{notes/}{exact_case.tex}
\section{Approximate Case}
\section{Asymptotic Case}

\appendix
\chapter{CPTP写像の3つの表現の同値性の証明}
  
\begin{thebibliography}{99}
\bibitem{nielsen2010quantum}Michael A Nielsen and Isaac L Chuang. 
    \textit{Quantum computation and quantum information}.
    Cambridge university press, 2010.
\bibitem{SagawaEntropy}Takahiro Sagawa. 
    \textit{Entropy, Divergence, and Majorization in Classical and Quantum Thermodynamics}. 
    Springer Singapore, 2022.
\bibitem{SagawaSaizensen}沙川貴大
    「非平衡統計力学 -ゆらぎの熱力学から情報熱力学まで-」
    共立出版, 2022. 
\bibitem{PhysLabResource}渡邉開人
    「Physics Lab. 2023 熱力学とリソース」
    \url{https://event.phys.s.u-tokyo.ac.jp/physlab2023/pdf/sta-article01.pdf}
\end{thebibliography}
  
\end{document}